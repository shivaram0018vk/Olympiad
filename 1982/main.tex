\documentclass[12pt,-letter paper]{article}
\usepackage{siunitx}
\usepackage{setspace}
\usepackage{gensymb}
\usepackage{xcolor}
\usepackage{caption}
%\usepackage{subcaption}
\doublespacing
\singlespacing
\usepackage[none]{hyphenat}
\usepackage{amssymb}
\usepackage{relsize}
\usepackage[cmex10]{amsmath}
\usepackage{mathtools}
\usepackage{amsmath}
\usepackage{commath}
\usepackage{amsthm}
\interdisplaylinepenalty=2500
%\savesymbol{iint}
\usepackage{txfonts}
%\restoresymbol{TXF}{iint}
\usepackage{wasysym}
\usepackage{amsthm}
\usepackage{mathrsfs}
\usepackage{txfonts}
\let\vec\mathbf{}
\usepackage{stfloats}
\usepackage{float}
\usepackage{cite}
\usepackage{cases}
\usepackage{subfig}
%\usepackage{xtab}
\usepackage{longtable}
\usepackage{multirow}
%\usepackage{algorithm}
\usepackage{amssymb}
%\usepackage{algpseudocode}
\usepackage{enumitem}
\usepackage{mathtools}
%\usepackage{eenrc}
%\usepackage[framemethod=tikz]{mdframed}
\usepackage{listings}
%\usepackage{listings}
\usepackage[latin1]{inputenc}
%%\usepackage{color}{   
%%\usepackage{lscape}
\usepackage{textcomp}
\usepackage{titling}
\usepackage{hyperref}
%\usepackage{fulbigskip}   
\usepackage{tikz}
\usepackage{graphicx}
\lstset{
  frame=single,
    breaklines=true
    }
    \let\vec\mathbf{}
    \usepackage{enumitem}
    \usepackage{graphicx}
    \usepackage{siunitx}
    \let\vec\mathbf{}
    \usepackage{enumitem}
    \usepackage{graphicx}
    \usepackage{enumitem}
    \usepackage{tfrupee}
    \usepackage{amsmath}
    \usepackage{amssymb}
    \usepackage{mwe} % for blindtext and example-image-a in example
    \usepackage{wrapfig}
    \graphicspath{{figs/}}
    \providecommand{\mydet}[1]{\ensuremath{\begin{vmatrix}#1\end{vmatrix}}}
    \providecommand{\myvec}[1]{\ensuremath{\begin{bmatrix}#1\end{bmatrix}}}
    \providecommand{\cbrak}[1]{\ensuremath{\left\{#1\right\}}}
    \providecommand{\brak}[1]{\ensuremath{\left(#1\right)}}
    \begin{document}
    \begin{enumerate}
\subsection*{Twenty-third International Olympiad,1982}
\item . The function $f\brak{n}$ is defined for all positive integers $n$ and takes on non-negative integer values. Also, for all $m,n$ 
\begin{align*}f\brak{m + n} - f\brak{m} - f\brak{n} = 0  \brak{or} 1 \end{align*} 
 \begin{align*}f\brak{2} = 0  , f\brak{3} > 0,and  f\brak{9999} = 3333.\end{align*}
		Determine $f\brak{1982}.$
	\item A non-isosceles triangle $A_1 A_2 A_3$ is given with sides $a_1,a_2,a_3$ ($a_i$ is the side opposite $A_i$). For all $i = 1, 2, 3, M_i$ is the midpoint of side $a_i$ and $T_i$ is the point where the incircle touches side $a_i$. Denote by $S_i$ the reflection. of $T_i$ in the interior bisector of angle $A_i$. Prove that the lines $M_1,S_1$,$ M_2S_2$ and $M_3S_3$ are concurrent.
\item Consider the infinite sequences $\cbrak{x_n}$ of positive real numbers with following properties: \\
		$ x_{0}=1,$ and for  all  $i \geq 0, x_{i+1} \leq x_i.$ \\
\brak{a} Prove that for every such sequence, there is $n \geq 1$ such that
		    \begin{align*} \frac{x^{2}_{0}}{x_{1}}+ \frac{x^{2}_{1}}{x_{2}}+ ...+\frac{x^{2}_{n-1}}{x_{n}} \geq 3.999.\end{align*}
\brak{b} Find such a sequence for which
 \begin{align*} \frac{x^{2}_{0}}{x_{1}}+ \frac{x^{2}_{1}}{x_{2}}+ ...+\frac{x^{2}_{n-1}}{x_{n}}< 4.\end{align*}
	    \item Prove that if $n$ is a positive integer such that the equation. \begin{align*}x^3 - 3xy^2 + y^3 = n \end{align*}  has a solution in integers $\brak{x, y}$, then it has at least three such solutions. Show that the equation has no solutions in integers when $n = 2891.$ 
\item The diagonals $AC$ and $CE$ of the regular hexagon $ABCDEF$ are divided by the inner points $M$ and $N$, respectively, so that \begin{align*} \frac{AM}{AC}=\frac{CN}{CE}=r.
		    \end{align*}
 Determine r if $B,$ $M,$ and $N$ are collinear.
\item Let $S$ be a square with sides of length $100$, and let $L$ be a path with in $S$ which does not meet itself and which is composed of line segments $A_0A_1, A_1A_2,.... A_{n-1}A_1$ with $A_0 \neq A_n$. Suppose that for every point $P$ of the boundary of $S$ there is a point of $L$ at a distance from $P$ not greater than $1/2$. Prove that there are two points $X$ and $Y$ in  $\&$ such that the distance between $X$ and $Y$ is not greater than $1$, and the length of that part of $L$ which lies between $X$ and $Y$ is not smaller than $198$.
    \end{enumerate}
    \end{document}
