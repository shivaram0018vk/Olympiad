\documentclass[12pt,-letter paper]{article}
\usepackage{siunitx}
\usepackage{setspace}
\usepackage{gensymb}
\usepackage{xcolor}
\usepackage{caption}
%\usepackage{subcaption}
\doublespacing
\singlespacing
\usepackage[none]{hyphenat}
\usepackage{amssymb}
\usepackage{relsize}
\usepackage[cmex10]{amsmath}
\usepackage{mathtools}
\usepackage{amsmath}
\usepackage{commath}
\usepackage{amsthm}
\interdisplaylinepenalty=2500
%\savesymbol{iint}
\usepackage{txfonts}
%\restoresymbol{TXF}{iint}
\usepackage{wasysym}
\usepackage{amsthm}
\usepackage{mathrsfs}
\usepackage{txfonts}
\let\vec\mathbf{}
\usepackage{stfloats}
\usepackage{float}
\usepackage{cite}
\usepackage{cases}
\usepackage{subfig}
%\usepackage{xtab}
\usepackage{longtable}
\usepackage{multirow}
%\usepackage{algorithm}
\usepackage{amssymb}
%\usepackage{algpseudocode}
\usepackage{enumitem}
\usepackage{mathtools}
%\usepackage{eenrc}
%\usepackage[framemethod=tikz]{mdframed}
\usepackage{listings}
%\usepackage{listings}
\usepackage[latin1]{inputenc}
%%\usepackage{color}{   
%%\usepackage{lscape}
\usepackage{textcomp}
\usepackage{titling}
\usepackage{hyperref}
%\usepackage{fulbigskip}   
\usepackage{tikz}
\usepackage{graphicx}
\lstset{
  frame=single,
    breaklines=true
    }
    \let\vec\mathbf{}
    \usepackage{enumitem}
    \usepackage{graphicx}
    \usepackage{siunitx}
    \let\vec\mathbf{}
    \usepackage{enumitem}
    \usepackage{graphicx}
    \usepackage{enumitem}
    \usepackage{tfrupee}
    \usepackage{amsmath}
    \usepackage{amssymb}
    \usepackage{mwe} % for blindtext and example-image-a in example
    \usepackage{wrapfig}
    \graphicspath{{figs/}}
    \providecommand{\mydet}[1]{\ensuremath{\begin{vmatrix}#1\end{vmatrix}}}
    \providecommand{\myvec}[1]{\ensuremath{\begin{bmatrix}#1\end{bmatrix}}}
    \providecommand{\cbrak}[1]{\ensuremath{\left\{#1\right\}}}
    \providecommand{\brak}[1]{\ensuremath{\left(#1\right)}}
\begin{document}
\begin{enumerate}
		\subsection*{TWENTY-SECOND INTERNATIONAL OLYMPIAD,1981}
\item $P$ is a point inside a given triangle $ABC.D, E, F$ are the feet of the perpendiculars from $P$ to the lines $BC, CA, AB$ respectively. Find all $P$ for which \\ $\frac{BC}{PD}+\frac{CA}{PE}+\frac{AB}{PF}$ is least.
\item Let $1 \leq r \leq n$ and consider all subsets of $r$ elements of the set $\cbrak{1,2,..., n}$. Each of these subsets has a smallest  member. Let $F\brak{n,r}$ denote the arithmetic mean of these smallest numbers; prove that $F\brak{n,r}= \frac{n+1}{r+1}$
\item Determine the maximum value of $m^{3}+n^{3}$, where $m$ and $n$ are integers satisfying $m, n  \epsilon  \cbrak{1,2,..., 1981}$ and $\brak{n^{2}-mn-m^{2}}^{2}=1$
\item \brak{a} For which values of $n > 2$ is there a set of $n$ consecutive positive integers such that the largest number in the set is a divisor of the least common multiple of the remaining $n-1$ numbers?

	\brak{b}For which values of $n > 2$ is there exactly one set having the stated property?
\item Three congruent circles have a common point $O$ and lie inside a given triangle. Each circle touches a pair of sides of the triangle. Prove that the incenter and the circumcenter of the triangle and the point $O$  are collinear
\item The function $f\brak{x, y}$ satisfies

$\brak{1} f\brak{0, y} = y + 1,$ \\ 
$\brak{2} f\brak{x + 1, 0} = f\brak{x, 1},$ \\
$\brak{3} f(x + 1, y + 1) = f\brak{x, f\brak{x + 1, y}},$ \\
for all non-negative integers $x, y$. Determine $ f\brak{4,1981}$.
\end{enumerate}
\end{document}
